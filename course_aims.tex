\documentclass{memoir}[11pt,oneside,a4paper,openany]
\PassOptionsToPackage{svgnames}{xcolor}
\usepackage{amsmath}
\usepackage{amssymb}
\usepackage{siunitx}
\usepackage{mhchem}
\usepackage{nth}
\usepackage{wrapfig}
\usepackage{hyperref}
\usepackage{graphicx}
\usepackage{bm}
\usepackage{xspace}
\usepackage{booktabs}
\usepackage{tcolorbox}
\tcbuselibrary{skins,breakable}
\usetikzlibrary{shadings,shadows}

\newenvironment{myexampleblock}[1]{%
    \tcolorbox[beamer,%
    noparskip,breakable,
    colback=LightGreen,colframe=DarkGreen,%
    colbacklower=LimeGreen!75!LightGreen,%
    title=#1]}%
    {\endtcolorbox}

\newenvironment{myalertblock}[1]{%
    \tcolorbox[beamer,%
    noparskip,breakable,
    colback=LightCoral,colframe=DarkRed,%
    colbacklower=Tomato!75!LightCoral,%
    title=#1]}%
    {\endtcolorbox}

\newenvironment{myblock}[1]{%
    \tcolorbox[beamer,%
    noparskip,breakable,
    colback=LightBlue,colframe=DarkBlue,%
    colbacklower=DarkBlue!75!LightBlue,%
    title=#1]}%
    {\endtcolorbox}


\setlrmarginsandblock{2cm}{2cm}{*}
\setulmarginsandblock{2cm}{*}{1}
\checkandfixthelayout

\setlength{\parskip}{0.3cm}

\renewcommand{\thefootnote}{\fnsymbol{footnote}}

\begin{document}
\chapter*{Course Aims}
After completing each chapter of this course, you should be able to do the following.
\section*{Chapter 1 - Origins of QM}
\begin{itemize}
	\item Describe and explain some of the scientific discoveries that led to the breakdown of classical mechanics in the late \nth{19} century.
	\item Describe what a \textbf{wavefunction} is, and explain how they can be calculated by solving the \textbf{Schr{\"o}dinger Equation.} Describe what makes a wavefunction acceptable or not.
	\item Qualitatively explain the \textbf{Born Interpretation} of the wavefunction, and explain how it allows wavefunctions to be \textbf{normalised}. Be able to normalise simple wavefunctions via integration.
	\item Explain \textbf{operators and observables}, and how \textbf{observables} correspond to \textbf{operators}. Calculate observables from operators and simple wavefunctions.
	\item Explain why the \textbf{Hamiltonian Operator} is special, and explain what observable it calculates.
	\item Show that the Schr{\"o}dinger Equation is a kind of \textbf{eigenvalue equation}. Be able to show that a wavefunction is an eigenfunction of a specific operator algebraically.
	\item Give the expression for the \textbf{expectation value} of an operator, and apply it to simple examples to calculate expectation values.
	\item Qualitatively explain \textbf{The Uncertainty Principle} in terms of position and momentum. 
	\item Explain the significance of the \textbf{commutator} in quantum mechanics.
	\item Summarise the five \textbf{Postulates of Quantum Mechanics} given at the end of this chapter.
\end{itemize}
\section*{Chapter 2 - Translational Motion}
\begin{itemize}
	\item Describe the meaning of \textbf{translational motion} and give the \textbf{Hamiltonian} for a particle free to move in one dimension.
	\item Using a given wavefunction and Hamiltonian, be able to calculate the energy eigenvalue for a particle free to move in one dimension. Explain the significance of the quantity $k\hbar$.
	\item Qualitatively explain the concept of \textbf{quantisation of energy}, and explain how and when it arises.
	\item Being given all relevant wavefunctions and Hamiltonians, be able to derive an expression for the energy of a particle confined to a one dimensional box. Show algebraically why this energy is \textbf{quantised}.
	\item Explain the significance of the \textbf{quantum number} n.
	\item Sketch the first three wavefunctions for a particle in a box and explain the trend in energies using wavelength as an illustration. Explain how the spacing between adjacent levels changes as the quantum number increases.
	\item Explain the concept of the \textbf{zero-point energy} in quantum mechanics.
	\item Give a qualitative description of the \textbf{Quantum-Classical Correspondence Principle}, and illustrate the description with the probability distributions for a particle in a box at various values of $n$. 
\end{itemize}
\section*{Chapter 3 - Vibrational Motion}
\begin{itemize}
	\item Explain what a \textbf{harmonic oscillator} is using classical mechanics. Understand \textbf{Hooke's Law}, including the significance of the force constant, $k$.
	\item Give an expression for the potential energy $V$, for a particle trapped in a \textbf{harmonic well}. Sketch how changing the value of $k$ will affect the resulting potential.
	\item Give the Schr{\"o}dinger Equation for a \textbf{quantum harmonic oscillator}. Quote the expression for the \textbf{vibrational energy levels}. Explain the significance of $\omega$, with reference to both $k$ and $m$. 
	\item Sketch the first few vibrational energy levels in a sketch of a harmonic potential. Explain how the \textbf{level spacing} differs from a particle in a one dimensional box.
	\item Identify the \textbf{normalisation constant}, \textbf{Hermite Polynomial}, and \textbf{Gaussian Function} in a given general vibrational wavefunction. 
	\item Sketch the first few vibrational wavefunctions and their probability distributions, using the sketches to illustrate both \textbf{the Quantum-Classical Correspondence Principle} and the \textbf{zero-point energy.}
	\item Explain the vibrational zero point energy in terms of a \textbf{classical pendulum}.
\end{itemize}
\section*{Chapter 4 - Rotational Motion}
\begin{itemize}
	\item Define \textbf{angular momentum}, \textbf{angular velocity}, and \textbf{moment of inertia} in classical terms. Contrast them to their linear analogues. Explain how angular momentum depends on moment of inertia.
	\item Give expressions for \textbf{rotational and translational kinetic energy}, in terms of both \textbf{velocity} and \textbf{momentum}.
	\item Sketch the \textbf{particle on a ring} system, and explain how the Hamiltonian is constructed in comparison to the translational Hamiltonian. 
	\item Verify that given wavefunctions are eigenfunctions of the given Hamiltonian, and produce the energy eigenvalues. Explain the significance of the quantity $m_l\hbar$ in comparison to $k\hbar$ from translations.
	\item Explain qualitatively the meaning of the \textbf{cyclic boundary condition}. Show mathematically that this leads to \textbf{quantised angular momentum}.
	\item Explain the additional complexities that arise when moving from a \textbf{particle on a ring} to a \textbf{particle on a sphere}. 
	\item State that the wavefunctions for the particle on a sphere are given by the \textbf{Spherical Harmonics}. Explain the significance of the spherical harmonics in the context of \textbf{atomic orbitals}.
	\item Explain the physical meanings of $l$ and $m_l$. Be able to give a reasonable sketch of \textbf{the vector model}, and explain how it demonstrates \textbf{quantised angular momentum}. 
	\item State the allowed values for $l$ and $m_l$, and reflect on this in terms of \textbf{atomic orbitals}.
	\item Give the expression for the energies of a particle on a sphere, and comment on this in relation to known expressions from rotational spectroscopy.
\end{itemize}
\section*{Chapter 5 - The Born-Oppenheimer Approximation}
\begin{itemize}
	\item Outline why \textbf{The Born-Oppenheimer Approximation} is necessary in chemistry, and explain what it is in terms of the separability of nuclear and electronic wavefunctions.
	\item Explain in what circumstances we are able to invoke the Born-Oppenheimer Approximation, with a discussion of different \textbf{timescales} and \textbf{energies}. 
	\item Describe how the Born-Oppenheimer Approximation allows molecular potential energy curves to be calculated.
	\item Explain why, as spectroscopists, we are able to \textbf{extend} the Born-Oppenheimer Approximation to encompass all different modes of molecular motion (translation, rotation, vibration, and electronic). 
	\item State how this allows us to separate the \textbf{total molecular energy} into \textbf{constituent parts} from each type of molecular motion. Explain why this is useful for physical chemists.
\end{itemize}
\section*{Chapter 6 - Rotational and Vibrational Quantisation of Molecules}
\begin{itemize}
	\item Explain how spectroscopy can provide information about molecules by the \textbf{interaction of molecules and EM radiation}.
	\item State the \textbf{Bohr Frequency Condition}, and explain all the terms in the equation.
	\item State the expression for \textbf{rotational energy} in \textbf{wavenumbers}, and calculate \textbf{rotational constants} with given data.
	\item Derive the expression for the \textbf{rotational line spacing}, and explain how it changes as $J$ increases.
	\item State the \textbf{gross} and \textbf{specific} rotational \textbf{selection rules}. Qualitatively explain their origins.
	\item Explain how the concept of the \textbf{reduced mass} allows diatomic molecules to be modelled as a simple \textbf{mass on a spring}. Calculate reduced masses using given data.
	\item State the expression for \textbf{vibrational energy} in \textbf{wavenumbers}, and calculate the value of $\tilde{v}$ using given data.
	\item Qualitatively explain the \textbf{limitations of the harmonic oscillator model}, and explain how using \textbf{anharmonicity} can get round them.
	\item Sketch a \textbf{harmonic potential} and a \textbf{Morse potential}, showing the key differences, and explaining why a \textbf{Morse potential} is more accurate for molecules.
	\item State the \textbf{gross} and \textbf{specific} vibrational \textbf{selection rules}. Explain qualitatively where they come from.
\end{itemize}




\end{document}
